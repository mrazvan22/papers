\documentclass[10pt,xcolor=table,aspectratio=169]{beamer}

\usepackage{graphicx}
\usepackage{caption}
\usepackage{subcaption}
\usepackage{transparent}
\usepackage{epstopdf} %converting to PDF
\usepackage{multicol} 
\usepackage{animate}[2017/05/18]
% \usepackage{subfig}

\usepackage{makecell}

% \usepackage{pdfx}
 
% \usepackage[utf8]{inputenc}
% \usepackage[T1]{fontenc}
\usepackage[table]{xcolor}    % loads also »colortbl« 
%  \usepackage{enumitem}
% \usepackage{ucltemplate}
\usepackage{color}

\usepackage{comment}

\usepackage{tabularx} % make width of table columns evenly distributed (see http://tex.stackexchange.com/questions/60601/evenly-distributing-column-widths)
% \newcolumntype{Y}{>{\centering\arraybackslash}X}

% make entire row bold or italic in table
\newcommand\setrow[1]{\gdef\rowmac{#1}#1\ignorespaces}
\newcommand\clearrow{\global\let\rowmac\relax}
\clearrow


\usepackage{amssymb}% http://ctan.org/pkg/amssymb
\usepackage{pifont}% http://ctan.org/pkg/pifont
\newcommand{\cmark}{\ding{51}}%
\newcommand{\xmark}{\ding{55}}%


%\usepackage{pgfgantt} % for grantt charts
\usepackage{rotating}
\usepackage[graphicx]{realboxes}
\usepackage[export]{adjustbox}
\usepackage{array}

\usepackage{rotating}
% \usepackage{tabularx, booktabs} % make width of table columns evenly distributed (see http://tex.stackexchange.com/questions/60601/evenly-distributing-column-widths)
% \newcolumntype{Y}{>{\centering\arraybackslash}X}

\DeclareMathOperator*{\argmin}{arg\,min}
\DeclareMathOperator*{\argmax}{arg\,max}

\usepackage{tikz}
\usetikzlibrary{arrows,positioning, shapes.symbols,shapes.callouts,patterns,shapes,chains,calc,backgrounds,fadings}

% \definecolor{parCol}{rgb}{0.1, 0.1, 1}
% \definecolor{stCol}{rgb}{0.1, 0.6, 0.1}
% \definecolor{bothCol}{rgb}{0, 0.5, 0.5}

\definecolor{parCol}{rgb}{0, 0, 0}
\definecolor{stCol}{rgb}{0, 0, 0}
\definecolor{bothCol}{rgb}{0, 0, 0}
\definecolor{blue3}{HTML}{86B7FC} % med blue
\definecolor{blue1}{HTML}{B5F1FF} % light blue
\definecolor{blue2}{HTML}{E0F9FF} % very light blue

\newcolumntype{C}[1]{>{\centering\let\newline\\\arraybackslash\hspace{0pt}}m{#1}}

\setlength{\tabcolsep}{0.2em}

 
 %% OVERVIEW OF WORK SO FAR %%
 
%Information to be included in the title page:
\title{BrainPainter: A software for the visualisation of brain structures, biomarkers and associated pathological processes}
\author[Raz]{\footnotesize{R\u{a}zvan V. Marinescu\inst{1,2} \and Arman Eshaghi\inst{2} \and Daniel C. Alexander\inst{2} \and Polina Golland\inst{1}}}



\institute{
1. Computer Science and Artificial Intelligence Laboratory, MIT, Cambridge, USA\\
% \and
2. Centre for Medical Image Computing, UCL, London, UK\\
3. Queen Square MS Centre, UCL Institute of Neurology, London, UK\\
}


\vspace{0em}
% \small{Centre for Medical Image Computing, University College London, UK}
% }

\date{}

\newcommand{\heightLogo}{0.8cm}

% logo of my university
\titlegraphic{
    \vspace{-2em}
%     \begin{figure}
%     \includegraphics[height=1.cm]{tadpole_logo}
%     \end{figure}


   \begin{figure}
   \begin{subfigure}{0.32\textwidth}
   \centering
%    2.\\
   \includegraphics[height=\heightLogo]{images/MIT_logo}
   \end{subfigure}
   \begin{subfigure}{0.32\textwidth}
   \hspace{2.5em}
    \centering
%    1.\\
   \includegraphics[height=\heightLogo, trim=70 0 0 10, clip]{images/ucl_logo}
   \end{subfigure}
   \end{figure}

%    \vspace{1em}
%    \tiny{Slides available online: https://people.csail.mit.edu/razvan/talk/tadpoleMICCAI/tadpole\_pres.pdf}
}

\setbeamercolor{frametitle}{fg=black}
\setbeamercolor{author in head/foot}{fg=black, bg=white} 
\setbeamercolor{institute in head/foot}{fg=black, bg=white} 
\setbeamercolor{title in head/foot}{fg=black, bg=white}
\setbeamercolor{date in head/foot}{fg=black, bg=white}

\setbeamersize{text margin left=10pt,text margin right=10pt}
% \setbeamertemplate{frametitle}{
%     \vspace{0.9em}
%     \insertframetitle
% %     \vspace{-3em}
% }
\setbeamertemplate{frametitle}{%
    \vspace{0.5em}
    \usebeamerfont{frametitle}\insertframetitle%
    \vphantom{g}% To avoid fluctuations per frame
    %\hrule% Uncomment to see desired effect, without a full-width hrule
    \par% <-- added
    \hspace*{-\dimexpr0.5\paperwidth-0.5\textwidth}% <-- calculation of left margin width
    \rule[0.5\baselineskip]{\paperwidth}{0.4pt}%
}

\setbeamertemplate{footline}
{
  \vspace{-3em}
  \leavevmode%
   \rule{\paperwidth}{0.3pt}
  \hbox{%
  \begin{beamercolorbox}[wd=.3\paperwidth,ht=2.25ex,dp=1ex,center]{author in head/foot}%
    \usebeamerfont{author in head/foot}Razvan Marinescu
  \end{beamercolorbox}%
%   \begin{beamercolorbox}[wd=.2\paperwidth,ht=2.25ex,dp=1ex,center]{institute in head/foot}%
%     \usebeamerfont{institute in head/foot} razvan@csail.mit.edu
%   \end{beamercolorbox}%
  \begin{beamercolorbox}[wd=.3\paperwidth,ht=2.25ex,dp=1ex,center]{institute in head/foot}%
    \usebeamerfont{institute in head/foot} BrainPainter 
  \end{beamercolorbox}%
  \begin{beamercolorbox}[wd=.3\paperwidth,ht=2.25ex,dp=1ex,center]{title in head/foot}%
    \usebeamerfont{title in head/foot}https://brainpainter.csail.mit.edu/
  \end{beamercolorbox}%
  \begin{beamercolorbox}[wd=.10\paperwidth,ht=2.25ex,dp=1ex,right]{date in head/foot}%
    \usebeamerfont{date in head/foot}\insertshortdate{}\hspace*{2em}
    \insertframenumber{} / \inserttotalframenumber\hspace*{2ex}
  \end{beamercolorbox}}%
  \vskip0pt%
}

% \usepackage{beamerthemesplit}

\newcommand{\backupbegin}{
   \newcounter{finalframe}
   \setcounter{finalframe}{\value{framenumber}}
}
\newcommand{\backupend}{
   \setcounter{framenumber}{\value{finalframe}}
}


\makeatletter
\long\def\beamer@author[#1]#2{%
  \def\and{\tabularnewline}
  \def\insertauthor{\def\inst{\beamer@insttitle}\def\and{\tabularnewline}%
  \begin{tabular}{rl}#2\end{tabular}}%
  \def\beamer@shortauthor{#1}%
  \ifbeamer@autopdfinfo%
    \def\beamer@andstripped{}%
    \beamer@stripands#1 \and\relax
    {\let\inst=\@gobble\let\thanks=\@gobble\def\and{, }\hypersetup{pdfauthor={\beamer@andstripped}}}
  \fi%
}
\makeatother
\beamertemplatenavigationsymbolsempty
\setbeamertemplate{caption}[numbered]
\setbeamercolor{caption name}{fg=black}
\setbeamercolor{itemize item}{fg=black}
\setbeamercolor{itemize subitem}{fg=black}
\setbeamercolor{enumerate item}{fg=black}
\setbeamercolor{enumerate subitem}{fg=black}
\setbeamertemplate{enumerate item}[default]
\setbeamertemplate{enumerate subitem}

\setbeamertemplate{itemize item}[circle]
\setbeamertemplate{itemize subitem}[circle]
% \setbeamertemplate{itemize subsubitem}[default]

% \renewcommand\labelitemi{---}

\makeatletter
\let\@@magyar@captionfix\relax
\makeatother
\begin{document}
 
\section{Introduction}

 \frame{\titlepage}
 
\setbeamerfont{frametitle}{size=\large}

\newcommand{\upgradeReportLoc}{../../upgrade_report}
\newcommand{\epsrcPresLoc}{\upgradeReportLoc/epsrcPres}
\newcommand{\jointModellingDiseaseLoc}{../../jointModellingDisease}
\newcommand{\pcaLongPaperLoc}{../../PCA_long_paper}
\newcommand{\voxFld}{../../voxelwiseDPM}
\newcommand{\tadpoleFld}{/research/tadpole}
\newcommand{\diffEqModelFld}{../../diffEqModel}



\newcommand*{\pcaLongFigs}{\pcaLongPaperLoc/figures}


% \includeonlyframes{1-20}
%\includeonlyframes{current}



\newcommand{\ovHeight}{2cm}

% 
% % % TODO continue with overview, move into commands
% \newcommand{\ovEBM}{
% \begin{subfigure}{0.47\textwidth}
% \centering
% 1. Modelled progression of PCA and tAD\\
% (using existing methods)
% \includegraphics[height=\ovHeight]{ebm_thumb.png}
% \end{subfigure}
% }
% 
% \newcommand{\ovVWDPM}{
% \begin{subfigure}{0.47\textwidth}
% \centering
% % \vspace{2.8em}
% 2. Developed Novel Spatio-temporal Model \\ (DIVE)\\
% \includegraphics[height=\ovHeight]{\upgradeReportLoc/images/vwdpm/blend14_adniThavgFWHM0InithistCl3Pr0Ra1_VWDPMStd.png}
% \end{subfigure}
% }
% 
% 
% \newcommand{\ovDKT}{
% \begin{subfigure}{0.47\textwidth}
% \centering
% \vspace{2em}
% 3. Developed Novel Transfer Learning \\ method (DKT) \\
% \vspace{0.5em}
% \includegraphics[height=2.2cm]{\jointModellingDiseaseLoc/paper/figures/disease_knowledge_transfer.pdf}
% \end{subfigure}
% }
% 
% 
% \newcommand{\ovTadpole}{
% \begin{subfigure}{0.47\textwidth}
% \centering
% \vspace{-2em}
% 4. Organised TADPOLE Competition\\
% \vspace{1em}
% \includegraphics[height=1.2cm,valign=t]{\upgradeReportLoc/epsrcPres/tadpole} 
% \end{subfigure}
% }
% 
% \newcommand{\ovPainter}{
% \begin{subfigure}{\textwidth}
% \centering
% \vspace{0.5em}
% 5. Created BrainPainter software\\
% \includegraphics[height=1.5cm]{cortical-front_1}\includegraphics[height=1.5cm]{cortical-back_1}\includegraphics[height=1.5cm]{subcortical_1}
% \end{subfigure}
% }


\definecolor{light-gray}{gray}{0.6}



% Story: motivation, problems with current neuroimaging software (FS, slicer). 
% Need to perform animations
% 

% \begin{frame}[t]
%  \frametitle{Current brain visualisation software has several limitations}
% 
%  \definecolor{green1}{rgb}{0, 0.6, 0}
% \definecolor{red1}{rgb}{0.9, 0, 0}
% % \newcommand{\xmark}{\ding{55}}%
% 
%  
% \newcommand{\myyes}{\textcolor{green1}{\Large{\textbf{\checkmark}}}}
% \newcommand{\myno}{\textcolor{red1}{\Large{\xmark}}}
% 
% %  \vspace{-0.5cm}
% \begin{figure}
% \centering
% \begin{subfigure}[t]{0.3\textwidth}
% \centering
%  \begin{minipage}[t][3.5cm][t]{\textwidth}
%  \centering
% \myno requires specialised input files\\
% \includegraphics[width=0.5\textwidth]{images/inputFile} 
%  \end{minipage}
% 
% \end{subfigure}
% \begin{subfigure}[t]{0.3\textwidth}
% \centering
%  \begin{minipage}[t][3.5cm][t]{\textwidth}
%  \centering
% \myno cannot be automated for image generation\\
% \includegraphics[width=0.5\textwidth]{images/freeviewInterface.jpg}
%  \end{minipage}
% 
% 
% \end{subfigure}
% \begin{subfigure}[t]{0.3\textwidth}
% \centering
%  \begin{minipage}[t][3.5cm][t]{\textwidth}
%  \centering
% \myno volumetric images hard to visualise\\
% \vspace{1.5em}
% \includegraphics[width=\textwidth]{images/seeleyImages} 
% \vspace{1.0em}
%  \end{minipage}
% \end{subfigure}
% \end{figure}
%  
% \end{frame}
% 
% 
% 
% 
% \begin{frame}[t]
%  \frametitle{Aim: Develop a brain visualisation software that overcomes such limitations}
% 
%  \definecolor{green1}{rgb}{0, 0.6, 0}
% \definecolor{red1}{rgb}{0.9, 0, 0}
% % \newcommand{\xmark}{\ding{55}}%
% 
%  
% \newcommand{\myyes}{\textcolor{green1}{\Large{\textbf{\checkmark}}}}
% \newcommand{\myno}{\textcolor{red1}{\Large{\xmark}}}
% 
% %  \vspace{-0.5cm}
% \begin{figure}
% \centering
% \begin{subfigure}[t]{0.3\textwidth}
% \centering
%  \begin{minipage}[t][3.5cm][t]{\textwidth}
%  \centering
% \myno requires specialised input files\\
% \includegraphics[width=0.5\textwidth]{images/inputFile} 
%  \end{minipage}
% 
% 
% \myyes generic and simple input files
% 
% \begin{itemize}
%  \item generated by e.g. spreadsheet
%  \item list of colours, one for each ROI
% \end{itemize}
% 
% 
% % \begin{table}
% % \centering
% %  \fontsize{7}{8}\selectfont
% % \begin{tabular}{c | c c c}
% %  Biomarkers &  Hippocampus & Inferior & ...\\
% %  &              & temporal & ...\\
% %   \hline
% % % Brain 1 & 0.6 & 2.3 & .. \\
% % % Brain 2 & 1.2 & 0.0 & .. \\
% % ... & \multicolumn{3}{c}{...}\\
% % \end{tabular}
% % \end{table}
% 
% \end{subfigure}
% \begin{subfigure}[t]{0.3\textwidth}
% \centering
%  \begin{minipage}[t][3.5cm][t]{\textwidth}
%  \centering
% \myno cannot be automated for image generation\\
% \includegraphics[width=0.5\textwidth]{images/freeviewInterface.jpg}
%  \end{minipage}
% 
% 
% \myyes can be used to generate multiple images
% 
% \includegraphics[height=1cm,trim=0 0 900 0]{images/brainTransparent} 
% \includegraphics[height=1cm,trim=0 0 900 0]{images/brainTransparent} 
% \includegraphics[height=1cm,trim=0 0 900 0]{images/brainTransparent} 
% \includegraphics[height=1cm]{images/brainTransparent} 
% 
% 
% 
% \end{subfigure}
% \begin{subfigure}[t]{0.3\textwidth}
% \centering
%  \begin{minipage}[t][3.5cm][t]{\textwidth}
%  \centering
% \myno volumetric images hard to visualise\\
% \vspace{1.5em}
% \includegraphics[width=\textwidth]{images/seeleyImages} 
% \vspace{1.0em}
%  \end{minipage}
% 
% 
% \myyes brain surface easy to visualise\\
% \includegraphics[width=0.5\textwidth]{images/cortical-outer_0} 
% 
% 
% \end{subfigure}
% 
% \end{figure}
%  
% \end{frame}

\begin{frame}[t]{Pathology progression in Alzheimer's disease}

\vspace{-2em}
\begin{figure}
\centering
% \newcommand{\speed}{10}
% \begin{animateinline}[autoplay,loop]{\speed}
%   \multiframe{1}{i=100+1}{% loop through pictures
% %   \multiframe{1}{i=1+1}{% loop through pictures \multiframe{nrOfPics}{i=initialVal+increment} 
%   \parbox{\textwidth}{
  \centering
  \begin{subfigure}[b]{1\textwidth}
   \centering
   \includegraphics[width=\textwidth,trim=0 30 0 70,clip]{images/dian-pib100.png}
  \end{subfigure}
  ~
%   \begin{subfigure}[b]{0.31\textwidth}
%    \centering
%   \includegraphics[width=\textwidth]{images/sara_video/inner-\i.png}
%   \end{subfigure}
%   ~
%   \begin{subfigure}[b]{0.31\textwidth}
%    \centering
%     \includegraphics[width=\textwidth]{images/sara_video/subcortical-\i.png}
%   \end{subfigure}
%   }  
%   }
% \end{animateinline}
Benzinger et al., PNAS, 2013
\end{figure}


\end{frame}


\begin{frame}[t]{Creating a movie showing pathology progression is difficult}
 
 One needs to:
\begin{itemize}
 \item Create special input files for neuroimaging software (Freesurfer, SPM, 3D slicer, etc ...)
 \item Need to interface via API, or launch GUI for every frame
 \item Find the right parameters:
\begin{itemize}
  \item viewing angles
  \item color thresholds
  \item etc ...
\end{itemize}
 \item Read documentation ...

 \end{itemize}

\onslide<2-> \textbf{\textcolor{red}{Could easily take 1 week!!!}}

\vspace{2em}
\onslide<3-> \textbf{\textcolor{black}{BrainPainter can create the movie in 30 minutes.}}
 

 
\end{frame}


% \begin{frame}
%  \frametitle{Further motivation}
% 
%  
% \newcommand{\speed}{2} 
% \newcommand{\animOne}{
% \begin{animateinline}[autoplay,loop]{\speed}  
% %    \multiframe{24}{i=1+1}{% loop through pictures
%   \multiframe{1}{i=1+1}{% loop through pictures \multiframe{nrOfPics}{i=initialVal+increment} 
%   \centering
%    \includegraphics[width=0.25\textwidth]{images/sara_video/outer-\i.png}
%   }
% \end{animateinline}
% }
%  
% \begin{itemize}
%  \item simple input files $\rightarrow$ no need to read specifications and integrate with Freesurfer, SPM, etc ..
%  \item automated image generation $\rightarrow$ create movies showing e.g. progression of brain pathology
% 
%  
% \begin{figure}
% \begin{tikzpicture}
% 
%  \node (A) at (-3,0) {
% \includegraphics[height=2cm,trim=0 0 900 0]{images/brainTransparent} 
% \includegraphics[height=2cm,trim=0 0 900 0]{images/brainTransparent} 
% \includegraphics[height=2cm,trim=0 0 900 0]{images/brainTransparent} 
% \includegraphics[height=2cm]{images/brainTransparent} 
% };
%  \node (B) at (3,0) {\animOne};
% 
%  \draw[thick,->] (A.east) -- (B.west);
%   \end{tikzpicture}
% 
% \end{figure}
% \end{itemize}
% 
% \vspace{-1em}
% 
% \begin{columns}
%  \begin{column}{0.5\textwidth}
%  \begin{itemize}
%    \item surface representation $\rightarrow$ easy visualisation of subcortical structures
%  \end{itemize}
%  \end{column}
% \begin{column}{0.5\textwidth}
% %   \begin{figure}
% %   \centering
%   \includegraphics[height=3cm]{images/DK_output/Image_1_subcortical}
% %  \end{figure}
% \end{column}
% \end{columns}
%  
%  
%  
% %  \includegraphics[height=2cm]{images/DK_output/Image_1_subcortical}
%  
%  
%  
% % 1. no need to integrate with Freesurfer, SPM, etc ... - Generic input data
% % 2. creating movies showing e.g. progression of pathology
% % 3. no need to show multiple slices due to surface visualisation
% \end{frame}

\begin{frame}{Towards a brain visualisation software that is easy to use}

\vspace{-1em}
 Need to simplify input data, which is voxelwise!
 
 \begin{columns}[t]
 \begin{column}[t]{0.5\textwidth}
%  \vspace{-7em}
  \begin{itemize}
 \item Idea: Color entire regions based on atlas, instead of voxelwise patterns
 \vspace{1em}
 \includegraphics[height=2.5cm]{images/destrieux}

  \item This enables us to provide the input as a list of colours for each region:

  \vspace{0.5em}
    \includegraphics[height=1.5cm]{images/BrainPainter_gradient_colours_simple}
%    \begin{tabular}{|c | c | c | c|}
%    \hline
%    \small
%    Hippocampus & Superior Parietal & Inferior Occipital & ...\\
%    \hline
%    (R,G,B) & (R,G,B) & (R,G,B)  & ... \\
%    \hline
%   \end{tabular}


\end{itemize}
 \end{column}
\onslide<2->  \begin{column}[t]{0.5\textwidth}

\begin{itemize}
  \item Even better: use a gradient of colours
   \vspace{3em}
  \includegraphics[height=3.5cm]{images/BrainPainter_gradient_colours}

  \item Can now generate input file from a spreadsheet program (MS Excel)
  
  
  \end{itemize}
     
     
     
  \end{column}

 
 \end{columns}

 




 
 
\end{frame}



\begin{frame}
 \frametitle{BrainPainter: How it works}

  
\begin{figure}
\centering
\includegraphics[height=5cm]{images/diagram.png}
\end{figure}
 
\begin{itemize}
 \item Uses Blender to generate images from pre-defined templates
\end{itemize}


 
\end{frame} 


\begin{frame}[t]{BrainPainter is customisable}

%  \begin{itemize}
%   \item Generates a brain visualisation from a list of numbers mapping to a color gradient
%   \item Customisable:


\begin{columns}[t]
\begin{column}{0.5\textwidth}


 \begin{itemize}
   \item supports three atlases:
   \begin{figure}
    \fontsize{8}{10}\selectfont
     \begin{subfigure}{0.27\textwidth}
      \centering
      Desikan-Killiany\\
      \includegraphics[height=1.5cm]{images/DK_output/Image_2_cortical-outer.png}
     \end{subfigure}
     \begin{subfigure}{0.27\textwidth}
      \centering
      Destrieux\\
      \includegraphics[height=1.5cm]{images/Destrieux_output/Image_2_cortical-outer.png}
     \end{subfigure}
     \begin{subfigure}{0.27\textwidth}
      \centering
      Tourville\\
      \includegraphics[height=1.5cm]{images/Tourville_output/Image_2_cortical-outer.png}
     \end{subfigure}

   \end{figure}
   
   \vspace{2em}
   
   \item supports different surfaces
   \begin{figure}
    \fontsize{8}{10}\selectfont
     \begin{subfigure}{0.27\textwidth}
      \centering
      pial\\
      \includegraphics[height=1.5cm]{images/DK_output/Image_2_cortical-outer.png}
     \end{subfigure}
     \begin{subfigure}{0.27\textwidth}
      \centering
      inflated\\
      \includegraphics[height=1.5cm]{images/DK_output_inflated/Image_2_cortical-outer.png}
     \end{subfigure}
     \begin{subfigure}{0.27\textwidth}
      \centering
      white matter\\
      \includegraphics[height=1.5cm]{images/DK_output_white/Image_2_cortical-outer.png}
     \end{subfigure}

   \end{figure}
   
   

\end{itemize}
 

\end{column}
\begin{column}{0.5\textwidth}  %%<--- here


 \begin{itemize}
   \item supports three pre-defined viewpoints
      \begin{figure}
    \fontsize{8}{10}\selectfont
     \begin{subfigure}{0.25\textwidth}
      \centering
      Outer Cortical\\
      \includegraphics[height=1.5cm]{images/DK_output/Image_2_cortical-outer.png}
     \end{subfigure}
     \begin{subfigure}{0.25\textwidth}
      \centering
      Inner Cortical\\
      \includegraphics[height=1.5cm]{images/DK_output/Image_2_cortical-inner.png}
     \end{subfigure}
     \begin{subfigure}{0.25\textwidth}
      \centering
      Subcortical\\
      \includegraphics[height=1.5cm]{images/DK_output/Image_2_subcortical.png}
     \end{subfigure}

   \end{figure}
   
   \vspace{2em}
   
      
   
   \item user-defined colour gradient
   \item resolution
   \item background color
   \item ...
   
%    \item useful pre-defined settings instead of full customisation
   

  \end{itemize}



\end{column}
\end{columns}

  
% \end{itemize}

\end{frame}
 
\begin{frame}
 \frametitle{Example Use 1: Visualise degree of atrophy in Alzheimer's disease}
 
\begin{figure}
\centering
 \includegraphics[width=0.5\textwidth]{images/young_3brains.png}
 
 Young et al, Nature Comms., 2018
\end{figure}
 
\end{frame}

\begin{frame}
 \frametitle{Example Use 2: Visualise temporal progression of atrophy in Alzheimer's subtypes}

 \begin{figure}
\centering
 \includegraphics[width=0.5\textwidth]{images/young_progression.png}
 
 Young et al, Nature Comms., 2018
\end{figure}
 
\end{frame}

\begin{frame}
 \frametitle{Example Use 3: Visualise subcortical atrophy in Huntington's disease}
\begin{figure}[htp]
\centering
% \subfigure{Stage 0\includegraphics[width=0.2\textwidth]{images/ebmhd_pngs/subcortical_stage0.png}}
% \subfloat[Stage 0]{
\begin{subfigure}{0.2\textwidth}
\centering
No disease
\includegraphics[width=1\textwidth,trim=70 0 70 0, clip]{images/ebmhd_pngs/subcortical_stage0.png}
\end{subfigure}
\begin{subfigure}{0.2\textwidth}
\centering
Early stage
\includegraphics[width=1\textwidth,trim=70 0 70 0, clip]{images/ebmhd_pngs/subcortical_stage3.png}
\end{subfigure}
\begin{subfigure}{0.2\textwidth}
\centering
Middle stage
\includegraphics[width=1\textwidth,trim=70 0 70 0, clip]{images/ebmhd_pngs/subcortical_stage6.png}
\end{subfigure}
\begin{subfigure}{0.2\textwidth}
\centering
Late stage
\includegraphics[width=1\textwidth,trim=70 0 70 0, clip]{images/ebmhd_pngs/subcortical_stage10.png}
\end{subfigure}\\
Wijeratne et al., Ann. Clin. Neurol., 2018
\end{figure}
 
\end{frame}

\begin{frame}
 \frametitle{Example Use 4: Animate the progression of amyloid spread in Alzheimer's disease}

 
\begin{figure}
\centering
\newcommand{\speed}{2} 
\begin{animateinline}[autoplay,loop]{\speed}  
    \multiframe{24}{i=1+1}{% loop through pictures
 % \multiframe{1}{i=1+1}{% loop through pictures \multiframe{nrOfPics}{i=initialVal+increment} 
  \parbox{\textwidth}{
  \centering
  \begin{subfigure}[b]{0.31\textwidth}
   \centering
   \includegraphics[width=\textwidth]{images/sara_video/outer-\i.png}
  \end{subfigure} 
  ~
  \begin{subfigure}[b]{0.31\textwidth}
   \centering
  \includegraphics[width=\textwidth]{images/sara_video/inner-\i.png}
  \end{subfigure}
  ~
  \begin{subfigure}[b]{0.31\textwidth}
   \centering
    \includegraphics[width=\textwidth]{images/sara_video/subcortical-\i.png} 
  \end{subfigure}
  }  
  }
\end{animateinline}

Garbarino and Lorenzi, IPMI, 2019
\end{figure}

  
\end{frame}


\begin{frame}
 \frametitle{BrainPainter runs straight from the browser - Live Demo}
 
 \begin{itemize}
  \item https://brainpainter.csail.mit.edu/
 
 \begin{figure}
  \includegraphics[width=0.6\textwidth]{images/frontPage}
 \end{figure}
 
 \item can also run from source: https://github.com/mrazvan22/brain-coloring
 \begin{itemize}
  \item requires no installation, run straight from docker container
 \end{itemize}
 
 \end{itemize}
 
  
\end{frame}

\begin{frame}
 \frametitle{Conclusion}

\begin{itemize}
 \item Created a software that allows easy generation of multiple brain images
 
 \vspace{1em}
 
 \item Runs from browser
 
 \vspace{1em}
 
 \item Enables quick movie creation
 
 \vspace{1em}
 
 \item Supports multiple atlases
 
 \vspace{1em}
 
 \item Useful pre-defined settings: viewpoints, surfaces, etc ...

 \vspace{1em}
 
 \item Turned Blender, a powerful graphics software, into a brain visualisation tool for neuroscience

 
\end{itemize}
 
 
 
\end{frame} 

\begin{frame}
 \frametitle{Future work}

 \vspace{-2em}
 
 \begin{columns}
  \begin{column}[t]{0.5\textwidth}
  \begin{itemize}
  \item improve robustness of website, add error messages for wrong input
  
  \vspace{2em}
  
  \item support for other brain templates: e.g. infants, mice\\
  \vspace{1em}
  \includegraphics[height=2cm]{images/infantBrain} \includegraphics[height=2cm]{images/mouse}
    
   \end{itemize}
  \end{column}
  \begin{column}[t]{0.5\textwidth}

  \begin{itemize}
  \item more atlases: e.g. Hammers
  
  \vspace{3em}
  
  \item other visualisations: e.g. white-matter tracts\\
  \vspace{2em}
  
  \includegraphics[height=2cm]{images/wmTracts}

%    \item 
  \end{itemize}

  
  \end{column}

 \end{columns}

\vspace{2em}
\begin{itemize}
  \item Blender allows much more: e.g. animating spread of toxic proteins through particle dynamics, fly-overs, ...
\end{itemize}
 
 
 
  
\end{frame}


\begin{frame}
 \frametitle{Acknowledgements}


% \begin{columns}
% \begin{column}{0.5\textwidth}

\textbf{Collaborators}
\begin{figure}
\begin{subfigure}{0.2\textwidth}
\centering
Polina Golland\\

\includegraphics[height=2cm]{images/polina} 
\end{subfigure}
\begin{subfigure}{0.2\textwidth}
\centering
Daniel Alexander
\includegraphics[height=2cm]{images/danny} 
\end{subfigure}
\begin{subfigure}{0.2\textwidth}
\centering
Arman Eshaghi\\
 
\includegraphics[height=2cm]{images/arman} 
\end{subfigure}
\begin{subfigure}{0.2\textwidth}
 \begin{itemize}
%   \item Polina Golland
%   \item Daniel Alexander
%   \item Arman Eshaghi
  \item Alexandra Young
  \item Sara Garbarino
  \item Peter Wijeratne
  \item ...
 \end{itemize}
\end{subfigure}



\end{figure}



% Collaborators:
\vspace{1em}

\textbf{Funders}\\
\begin{columns}
\begin{column}{0.4\textwidth}

\vspace{1em}

\includegraphics[height=2cm]{images/nac_logo} \hspace{2em}
\includegraphics[height=2cm]{images/epsrc_logo}
\end{column}
\begin{column}{0.4\textwidth}
% \vpsace{1em}
Anders Winkler for the 3D brain templates\\
 \begin{itemize}
  \item https://brainder.org/research/brain-for-blender/
 \end{itemize}

\end{column}

\end{columns}
 

 

  
\end{frame}


% \end{comment}

\end{document}



