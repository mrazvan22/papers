%% 
%% Copyright 2007, 2008, 2009 Elsevier Ltd
%% 
%% This file is part of the 'Elsarticle Bundle'.
%% ---------------------------------------------
%% 
%% It may be distributed under the conditions of the LaTeX Project Public
%% License, either version 1.2 of this license or (at your option) any
%% later version.  The latest version of this license is in
%%    http://www.latex-project.org/lppl.txt
%% and version 1.2 or later is part of all distributions of LaTeX
%% version 1999/12/01 or later.
%% 
%% The list of all files belonging to the 'Elsarticle Bundle' is
%% given in the file `manifest.txt'.
%% 
%% Template article for Elsevier's document class `elsarticle'
%% with harvard style bibliographic references
%% SP 2008/03/01

%\documentclass[preprint,12pt,authoryear]{elsarticle}

%% Use the option review to obtain double line spacing
%% \documentclass[authoryear,preprint,review,12pt]{elsarticle}

%% Use the options 1p,twocolumn; 3p; 3p,twocolumn; 5p; or 5p,twocolumn
%% for a journal layout:
%% \documentclass[final,1p,times,authoryear]{elsarticle}
%\documentclass[final,1p,times,twocolumn,authoryear]{elsarticle}
%% \documentclass[final,3p,times,authoryear]{elsarticle}
%\documentclass[final,3p,times,twocolumn,authoryear]{elsarticle}
%% \documentclass[final,5p,times,authoryear]{elsarticle}

\documentclass[final,5p,times,twocolumn,authoryear]{elsarticle}

%% For including figures, graphicx.sty has been loaded in
%% elsarticle.cls. If you prefer to use the old commands
%% please give \usepackage{epsfig}

%% The amssymb package provides various useful mathematical symbols
\usepackage{amssymb}
\usepackage{array}

\usepackage{graphicx}
\usepackage{placeins}

\usepackage{xpatch}
% \usepackage{biblatex}

\usepackage[font=normalfont,labelfont=bf]{caption}

%% The amsthm package provides extended theorem environments
%% \usepackage{amsthm}

%% The lineno packages adds line numbers. Start line numbering with
%% \begin{linenumbers}, end it with \end{linenumbers}. Or switch it on
%% for the whole article with \linenumbers.
%% \usepackage{lineno}



\usepackage{etoolbox}
% \patchcmd{<cmd>}{<search>}{<replace>}{<success>}{<failure>}
\patchcmd{\emailauthor}{(#2)}{}{}{}
\patchcmd{\urlauthor}{(#2)}{}{}{}

\journal{}

\makeatletter
\def\ps@pprintTitle{%
 \let\@oddhead\@empty
 \let\@evenhead\@empty
 \def\@oddfoot{}%
 \let\@evenfoot\@oddfoot}
\makeatother

\usepackage{xcolor}


\usepackage{hyperref}
\definecolor{links}{HTML}{01368e}
% \hypersetup{colorlinks,linkcolor=links,urlcolor=links}

\makeatletter
\hypersetup{colorlinks=true}
\AtBeginDocument{\@ifpackageloaded{hyperref}
  {\def\@linkcolor{links}
   \def\@anchorcolor{links}
   \def\@citecolor{links}
   \def\@filecolor{links}
   \def\@urlcolor{links}
   \def\@menucolor{links}
   \def\@pagecolor{links}
\begingroup
  \@makeother\`%
  \@makeother\=%
  \edef\x{%
    \edef\noexpand\x{%
      \endgroup
      \noexpand\toks@{%
        \catcode 96=\noexpand\the\catcode`\noexpand\`\relax
        \catcode 61=\noexpand\the\catcode`\noexpand\=\relax
      }%
    }%
    \noexpand\x
  }%
\x
\@makeother\`
\@makeother\=
}{}}
\makeatother


% \usepackage{etoolbox}
% \makeatletter
%     %replace first instance (first tnote)
%     \patchcmd{\fnmark}{\ding{73}}{\dag}{}{\@latex@error{Failed to path \string\fnmark\space for \string\ding{73}}}
%     %replace second instance (second tnote)
% %     \patchcmd{\tnotemark}{\ding{73}\ding{73}}{\dag\dag}{}{\@latex@error{Failed to path \string\tnotemark\space for \string\ding{73}\string\ding{73}}}
%     %replace first instance (first tnote)
%     \patchcmd{\fntext}{\ding{73}}{\dag}{}{\@latex@error{Failed to path \string\fntext\space for \string\ding{73}}}
%     %replace second instance (second tnote)
% %     \patchcmd{\tnotetext}{\ding{73}\ding{73}}{\dag\dag}{}{\@latex@error{Failed to path \string\tnotetext\space for \string\ding{73}\string\ding{73}}}
% \makeatother


\usepackage{filecontents}

\begin{filecontents*}{bibliography.bib}



\end{filecontents*}

\usepackage{natbib}


\begin{document}


\begin{frontmatter}

%% Title, authors and addresses

%% use the tnoteref command within \title for footnotes;
%% use the tnotetext command for theassociated footnote;
%% use the fnref command within \author or \address for footnotes;
%% use the fntext command for theassociated footnote;
%% use the corref command within \author for corresponding author footnotes;
%% use the cortext command for theassociated footnote;
%% use the ead command for the email address,
%% and the form \ead[url] for the home page:
%% \title{Title\tnoteref{label1}}
%% \tnotetext[label1]{}
%% \author{Name\corref{cor1}\fnref{label2}}
%% \ead{email address}
%% \ead[url]{home page}
%% \fntext[label2]{}
%% \cortext[cor1]{}
%% \address{Address\fnref{label3}}
%% \fntext[label3]{}

\title{BrainPainter: A software for the visualisation of brain structures and pathology}

%% use optional labels to link authors explicitly to addresses:
%% \author[label1,label2]{}
%% \address[label1]{}
%% \address[label2]{}

\ead{razvan@csail.mit.edu}
\ead[url]{https://github.com/mrazvan22/brain-coloring}


\address[ucl]{Centre for Medical Image Computing, University College London, Gower Street, London, United Kingdom, WC1E 6BT}
\address[mit]{Computer Science and Artificial Intelligence Laboratory, Massachusetts Institute of Technology, Cambridge, USA, MA 02139}

% \address[drc]{Dementia Research Centre, University College London Institute of Neurology, London, United Kingdom, WC1N 3AX}

\author[ucl,mit]{R\u{a}zvan V. Marinescu}
\author[ucl]{Daniel C. Alexander}
\author[mit]{Polina Golland}



% \address{University College London, Centre for Medical Image Computing, Gower Street, London, WC1E 6BT}


\begin{abstract}


\end{abstract}

\begin{keyword}
Alzheimer's disease \sep 
Brain visualisation
\end{keyword}

\end{frontmatter}

%% \linenumbers

%% main text
\FloatBarrier
\section{Introduction}
\label{intro}

% diagram showing the aim: input numbers and output images

Visualisation of brain structure, function and pathology is crucial for understanding the mechanisms underlying certain diseases.

Several tools are currently available for brain visualisation ...

We present BrainPainter, a software for easy visualisation of structures, pathology and biomarkers in the brain. As opposed to previous software, it automatically generates images from pre-defined view-points given a set of input numbers for each brain region used to define their color on a pre-defined color scale.  


%\FloatBarrier
\section{Design}
\label{design}

% diagram showing the Design. what kind of atlases can it take, type of colouring, blender integration


BrainPainter works 



\FloatBarrier
\section{Use case: Show progression of pathology}
\label{design}


\section{Conclusion} 




\FloatBarrier
\section{Acknowledgements}




RVM is supported by the EPSRC Centre For Doctoral Training in Medical Imaging with grant EP/L016478/1. NPO, FB, SK, and DCA are supported by EuroPOND, which is an EU Horizon 2020 project. ALY is currently supported by an EPSRC Doctoral Prize fellowship and was previously supported by EPSRC grant EP/J020990/01. DCA is supported by EPSRC grants J020990, M006093 and M020533. Data collection and sharing for this project was funded by the Alzheimer's Disease Neuroimaging Initiative (ADNI) (National Institutes of Health Grant U01 AG024904) and DOD ADNI (Department of Defense award number W81XWH-12-2-0012). FB is supported by the NIHR UCLH biomedical research centre and the AMYPAD project, which has received support from the EU-EFPIA Innovative Medicines Initiatives 2 Joint Undertaking (AMYPAD project, grant 115952). This project has received funding from the EU Horizon 2020 research and innovation programme under grant agreement No 666992.


% \appendix





%% The Appendices part is started with the command \appendix;
%% appendix sections are then done as normal sections
%% \appendix

%% \section{}
%% \label{}

%% If you have bibdatabase file and want bibtex to generate the
%% bibitems, please use
%%
%%  \bibliographystyle{elsarticle-harv} 
%%  \bibliography{<your bibdatabase>}

%% else use the following coding to input the bibitems directly in the
%% TeX file.

% \begin{thebibliography}{00}
% 
% \bibitem[Prince(2014)]{prince2014world}
% Prince, M. and Jackson, J., 2014. World Alzheimer Report 2009, Alzheimer's Disease International.
% 
% %% \bibitem[Author(year)]{label}
% %% Text of bibliographic item
% 
% \end{thebibliography}

%\section*{References}

\bibliographystyle{elsarticle-harv}
\bibliography{bibliography}

% \printbibliography
% 
% \xpatchbibmacro{date+extrayear}{%
%   \printtext[parens]%
% }{%
%   \setunit{\addperiod\space}%
%   \printtext%
% }{}{}
% 
% \printbibliography


\end{document}




\endinput


%%
%% End of file `elsarticle-template-harv.tex'.
